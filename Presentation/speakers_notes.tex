\documentclass[a4paper, 14pt]{extarticle}

%% Установка размеров страницы
\usepackage[left=30mm, top=20mm, right=10mm, bottom=20mm, nohead, nofoot]{geometry}
\geometry{papersize={210mm,297mm}}

%% Работа с русским языком
\usepackage{cmap} 		  			 % Поиск в pdf
\usepackage[T2A]{fontenc} 			 % Кодировка
\usepackage[utf8]{inputenc} 		 % Кодировка исходного текста
\usepackage[english, russian]{babel} % Локализация и переносы

%% Дополнительная работа с математикой
\usepackage{icomma} % "Умная" запятая: $0,2$ – число, а $0, 2$ – перечисление

%% Шрифты 
\usepackage{euscript} % Шрифт "Евклид"

%% Отступы в нумерованных списках
\usepackage{enumitem}


%% Заголовок
\title{<<Современные легковесные модели оценки карты глубины>>}
\author{Матвеев А.С.}
\date{\today}


\usepackage[onehalfspacing]{setspace} % Междустрочный интервал 1,5 строки (пакет)
\begin{document}
	\maketitle
	\thispagestyle{empty} % Удалить номер страницы
	\setstretch{1.5} 	  % Междустрочный интервал 1,5 строки (команда)
	%\setcounter{page}{2} % Начать нумерацию страниц с номера 2
	\pagebreak
	
	
	\setlength{\parindent}{1.25cm} % Абзацный отсуп 1,25 см
	\textit{[Слайд 1]} \textbf{Титульный лист}
	
	Добрый день, уважаемые коллеги! Вашему вниманию предоставляется доклад на тему <<Современные легковесные модели оценки карты глубины>> Матвеева Артёма Сергеевича. Соавторами работы являются Болбаков Роман Геннадьевич и Макаров Илья Андреевич.
	\bigskip
	
	
	\textit{[Слайд 2]} \textbf{Актуальность и новизна}
	
	Говоря об актуальности, можно выделить повсеместное применение карт глубины в ряде областей. Например, компьютерном зрении, робототехнике, виртуальной и дополненной реальностях и т.д. Легковесные модели позволяют решать задачи в условиях слабой доступности к большим вычислительным мощностям. 
	
	Говоря о новизне, можно подчеркнуть, что в данной обзорной работе представлены современные легковесные решения близкие к тяжеловесным моделям с точки зрения ряда метрик оценки качества.
	\bigskip
	
	
	\textit{[Слайд 3]} \textbf{Цель и задачи работы}
	
	Целью работы является выполнение обзора современных легковесных моделей оценки карты глубины, главным образом, для задач компьютерного зрения и робототехники. 
	
	Задачами работы является анализ трёх работ, опубликованных за последние пять лет. 
	\bigskip
	
	
	\textit{[Слайд 4]} \textbf{Постановка задачи. FastDepth}
	
	Рассмотрим первую работу, опубликованную в 2019 году сотрудниками массачусетского технологического института. На данном слайде можно видеть формальную постановку задачу. Представленная модель обучается с учителем. Каждый объект обучающей выборки -- это трёхканальное изображение размером $224 \times 224$. Ответ -- это карта глубины. \\ Мы хотим получить модель, которая выполняет отображение из множества пикселей исходного изображения в множество пикселей карты глубины. В качестве метрик используются RMSE и $\delta_1$ -- доля с порогом 1.25. 
	%\vspace{30mm}
	\bigskip
	
	
	\textit{[Слайд 5]} \textbf{Архитектура FastDepth}
	
	Архитектура состоит из энкодера, который представляет из себя модель MobileNet, декодера и также стоит отметить использование skip connections в декодере. Свёртка, выполняемая способом, представленным в MobileNet, значительно ускоряет работу модели. На слайде приведена математическая запись предлагаемого подхода. Проще говоря, мы разделяем входное трёхканальное изображение на 3 отдельных канала. Аналогично делаем с ядром свёртки. После этого применяем отдельно канал ядра к каналу изображения. Получаем 3 карты признаков после свёртки. Стекаем получившиеся карты признаков. Относим этот стек в ячейку итогового изображения. То есть получается изображение, у которого в ячейках записаны эти 3 карты признаков.
	\bigskip
	
	
	\textit{[Слайд 6]} \textbf{Особенности FastDepth}
	
	На данном слайде можно увидеть графическое представление архитектуры модели. В качестве особенностей FastDepth можно выделить использование TVM-компилятора для ускорения работы модели на конкретном железе, а также использование алгоритма NetAdapt для прунинга сети после обучения -- удаления избыточных каналов из карт признаков, что позволяет сделать модель ещё эффективнее. 
	\bigskip


	\textit{[Слайд 7]} \textbf{Результаты экспериментов. FastDepth (часть 1)}
	
	На данном слайде можно увидеть результаты модели по ряду метрик качества. Видно, что модель сильно выигрывает по количеству операций MAC (сложение и накопление), а также по времени работы на CPU и GPU. По метрикам RMSE и $\delta_1$ она также может посоревноваться с некоторыми тяжеловесными моделями. 
	\bigskip


	\textit{[Слайд 8]} \textbf{Результаты экспериментов. FastDepth (часть 2)}
	
	На данном слайде видны результаты с точки зрения оценки глубины на реальных изображениях из набора данных NYUDv2. (a) обозначает входное RGB-изображение, (b) ground truth, (c) их модель без skip connections и без прунинга, (d) их модель со skip connections и без прунинга, (e) их модель со skip connections и прунингом, (f) карта ошибок между (e) и (b) -- чем краснее, чем больше ошибок. 
	\bigskip
	
	
	\textit{[Слайд 9]} \textbf{Постановка задачи. Hu J. и др.}
	
	Рассмотрим вторую работу, опубликованную в 2021 году сотрудниками нескольких китайских университетов. На данном слайде можно видеть формальную постановку задачу. В целом, всё совпадает с предыдущей работой за исключением размеров изображений и применения ещё одной функции потерь на этапе внедрения парадигмы дистилляции знаний с вспомогательными данными.
	%\vspace{30mm}
	\bigskip
	
	
	\textit{[Слайд 10]} \textbf{Архитектура. Hu J. и др.}
	
	Архитектура состоит из 4 слоёв слияния и сжатия признаков. Каждый этот слой представляет из себя последовательность -- поканальный механизм внимания + свёртка. Далее идёт этап масштабирования карт признаков, конкатенация и свёртка. 
	\bigskip
	
	
	\textit{[Слайд 11]} \textbf{Особенности модели Hu J. и др.}
	
	Интересной особенностью является использование парадигмы дистилляции знаний. Этот подход помогает бороться с низкой точностью модели на этапе её работы. Для улучшения дистилляции знаний существуют разные подходы. Авторы выбрали использование вспомогательных данных. Так как оценка глубины -- это нелинейное отображение из пространства RGB в пространство глубины, то KD аппроксимирует эту функцию на основе данных. Чем лучше данные, тем лучше аппроксимация. \\ Также сбор вспомогательных данных достаточно легкая задача в данном случае. Из графика видно, что indoor сцены из разных наборов данных достаточно схожи по распределениям глубины.  
	\bigskip
	
	
	\textit{[Слайд 12]} \textbf{Результаты экспериментов. Hu J. и др. (часть 1)}
	
	На данном слайде можно увидеть результаты модели по ряду метрик качества. Видно, что модель обходит конкурентов по всем метрикам. 
	\bigskip
	
	
	\textit{[Слайд 13]} \textbf{Результаты экспериментов. Hu J. и др. (часть 2)}
	
	На данном слайде видны результаты с точки зрения оценки глубины на реальных изображениях из набора данных NYUDv2.
	\bigskip
	
	
	\textit{[Слайд 14]} \textbf{Постановка задачи. DANet}
	
	Рассмотрим третью работу, опубликованную в 2022 году сотрудниками пекинского университета. На данном слайде можно видеть формальную постановку задачи. Размерность изображений совпадает с предыдущей работой. Авторы рассматривают предсказание карты глубины следующим образом -- диапазон глубины всей сцены разбивается на $N$ ячеек. Для входного изображения $I$ необходимо предсказать центральные значения ячеек глубины $c$ и карты вероятности попадания каждого пикселя в соответствующую ячейку глубины $PХ$. В итоге пиксель карты глубины получается в результате линейной комбинации $c$ и $P$. 
	%\vspace{30mm}
	\bigskip
	
	
	\textit{[Слайд 15]} \textbf{Архитектура. DANet}
	
	Архитектура состоит из энкодера -- пирамидального трансформера и декодера.
	\bigskip
	
	
	\textit{[Слайд 16]} \textbf{Результаты экспериментов. DANet (часть 1)}
	
	На данном слайде можно увидеть результаты модели по ряду метрик качества. RMS -- это RMSE. Видно, что модель обходит ранее рассмотренные модели по RMSE и $\delta_1$.
	\bigskip
	
	
	\textit{[Слайд 17]} \textbf{Результаты экспериментов. DANet (часть 2)}
	
	На данном слайде видны результаты с точки зрения оценки глубины на реальных изображениях из набора данных NYUDv2 (первые 2 строки). Видно, что данная модель максимально совпадает по распределению глубины с ground truth. 
	\bigskip
	
	
	\textit{[Слайд 18]} \textbf{Заключение}
	
	В результате проделанной работы были выполнены все поставленные задачи.
	\bigskip
	
	\textit{[Слайд 19]} \textbf{Спасибо за внимание!}
	
\end{document}